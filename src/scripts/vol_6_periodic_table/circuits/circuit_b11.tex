\documentclass[border=10pt]{standalone}
\usepackage[siunitx]{circuitikz}
\usepackage{amsmath}
\usepackage{xcolor}

\begin{document}
\begin{circuitikz}[american, thick, scale=1.2]

% Custom Colors
\definecolor{corecolor}{HTML}{FF3366}
\definecolor{halocolor}{HTML}{99FFEE}
\definecolor{linkcolor}{HTML}{AAAAAA}

% The Alpha Core (Tight Inner Inductive Network)
\draw[color=corecolor] (0, 0) node[circle, fill, inner sep=2.5pt, label=-90:\textsf{Alpha Core (2p, 2n)}] (core) {};
\draw[color=corecolor] (core) ++(0,-0.6) node[] {\Large $\mathbf{L_{core}}$};
\draw[color=corecolor] (core) ++(0, 0.6) node[] {\Large $Q_{high}$};

% The 7-Nucleon Halo Outer Ring
\draw[color=halocolor, dashed, line width=1.5pt] (core) circle (4cm);

% Define the Halo Nodes (7 Nucleons)
\foreach \i in {1, 2, ..., 7} {
    \pgfmathsetmacro{\angle}{360/7 * (\i - 1) + 90}
    \draw[color=halocolor] (\angle:4cm) node[circle, fill, inner sep=2pt] (halo\i) {};
    \draw[color=halocolor] (\angle:4.5cm) node[] {$\mathbf{L_{halo\i}}$};
    
    % Draw mutual inductance coupling from Core to each Halo node
    \draw[color=linkcolor, <->, bend right=15] (core) to node[midway, fill=white, inner sep=1pt, sloped] {$M_{c-h}$} (halo\i);
}

% Draw extremely weak internal mutual coupling between adjacent halo nodes
\foreach \i in {1,2,3,4,5,6} {
    \pgfmathsetmacro{\nexti}{\i + 1}
    \draw[color=linkcolor, <->, dotted] (halo\i) to [bend right=20] (halo\nexti);
}
\draw[color=linkcolor, <->, dotted] (halo7) to [bend right=20] (halo1);

\node[color=linkcolor] at (-3.5, 3.5) {$M_{c-h} \ll L_{core}$};

% Orbital Electrons (5e-)
\draw[color=orange, dashed, line width=1.5pt] (core) circle (6cm);
\node[color=orange, align=center] at (0, -6.5) {\large \textbf{Orbital Shell ($5e^-$ Displacement Currents)}};
\draw[color=red, <->, dashed, thick] (0,-4) -- (0,-6) node[midway, right] {$M_{orbit}$};

\node[color=halocolor, align=center] at (0, -4.8) {\Large \textsf{Topological Halo Dispersion} \\[1ex] \textsf{Massive Parasitic Array}};

\end{circuitikz}
\end{document}
